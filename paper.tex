\documentstyle[12pt]{article}
\renewcommand{\baselinestretch}{1.7}
\textwidth 151mm \textheight 220mm \topmargin -10mm \oddsidemargin
2mm
\begin{document}
\begin{titlepage}
\title{Filamentation instability of dust-acoustic wave in a
collisional plasma with a variable charge}

\date{}
\maketitle

\begin{abstract}
The filamentation instability of dust-acoustic wave in a collisional
plasma with variable-charge dusts is studied using fluid theory. The
effect of ion drag force on the development of this instability is
investigated. It is shown that the filamentary mode appears at the
initial stages of void formation in dusty plasma, when the ion drag
force acting on the dust particles is stronger than the external
electric force. This instability is more easily excited when the
percentage of free electrons is reduced and the dust particles are
sufficiently large. Furthermore, the establishment time of the
filamentation structure and the instability development threshold
are obtained.
\end{abstract}
\end{titlepage}
\newpage

\vskip 1cm {\bf\large I. INTRODUCTION} \vskip 0.5cm

Dusty or complex plasmas are partially ionized gases composed of
neutral atoms, electrons, ions, and charged dust particles. Each
dust grain is a solid particulate matter that usually acquire a
large electric charge by collecting electrons and ions from the
plasma.$^{1-3}$ The dust-charge variation due to electron and ion
capture/release by the dust grains can affects various collective
phenomena in dusty plasmas. Recently a lot of theoretical and
experimental research has been conducted on the dust void and other
structures in dusty plasmas.$^{1-14}$

A dust void is a region without any dusty particle in the bulk of
the plasma and is the result of a balance of forces acting on the
dust particles. Samsonov and Goree$^{5}$ found that the spontaneous
formation of voids will be developed by a sudden onset of a
filamentary mode associated with increased local ionization which
attribute to a depletion of the dust number density. They found that
the boundary between void region and dusty cloud is typically quite
sharp. Morfill et al.$^{6}$ experimentally investigated the effect
of the plasma pressure on the void formation. They invoked the
thermophoretic forces (a neutral temperature gradient force) to
explain void formation. Thomas et al.$^{7}$ showed experimentally
the formation of voids around negatively biased probes in a dusty
plasma. The negative dust grains repel from the negatively biased
probe and a dust void would form around the probe. Goree et
al.$^{8}$ found that a minimum ionization rate is required to
sustain a stable void equilibria. In a region of reduced dust
density, a reduced depletion of the electrons by the dust particles
leads to a higher electron density and a consequently a higher
ionization rate. The dust density perturbation produces a positive
space charge with respect to the surrounding medium, which yields to
build up an electric field directed outward from the void center.
This electric field gives rise to an inward electric force and an
outward ion drag force on negatively charged dust particles which
tends to expel them from their position. Therefore, in equilibrium
state there is a balance between the outward ion drag force and the
inward electric force on the dust particles that results in
formation of dust-free region or void.$^{5,8}$

The ion-drag force in dusty plasmas arises from the ion orbital
motion around negatively charged dust particles as well as from the
momentum transfer from all the ions which are collected by the dust
grains.$^{1}$ The ion drag on the dust grains is important for the
instability and can lead to the formation of the void.$^{1,2,5-8}$
This force is proportional to the square of the particle
radius$^{1}$. Hence, when the dust grains size exceeds a critical
value, the outward ion drag force dominates over the inward
electrostatic force. Thus, the fluctuation will grow in the dusty
plasma and manifest itself as filamentation, which is valid only at
the initial stage of the instability. For a small particle size, the
electrostatic force will dominate, and the region of redused dust
density will be filled up once again by dust, and the filamentation
instability will disappear. Therefore, the threshold of the
filamentary mode development is dependent on the particle size and
the electric field strength.$^{2,8}$

Filamentation instability is one of the fundamental instabilities
that may arise in response of a plasma to an externally applied
electric field. Electron and/or ion beam moving through a plasma
will filament by a this instability that has a physical mechanism
closely related to the Weibel instability. The filamentation
instability convert the kinetic energy of  beam into the
electromagnetic energy. On the other hand, the filamentation
instability and magnetic field generation attracts great
attention.$^{14-25}$ Recently, we have investigated the
filamentation instability of dust-acoustic wave$^{24}$ ($DAW$) and
dust ion-acoustic wave$^{25}$ ($DIAW$) in a current-driven dusty
plasma by using the kinetic approach. In the present work, using
fluid theory we will investigate the filamentation instability of
$DAW$ at the initial stage of the void formation in dusty plasma. We
consider the dust charge fluctuation as well as ion drag on the
development of the filamentation instability at the dust-acoustic
time scale. We also determine the threshold of the filamentation
instability development and obtain the period of cross structure.

This work is organized into four sections. In Sec. II, the problem
is formulated and the basic set of equations is given.  In Sec. III,
the dispersion relation describing the dust-acoustic instability in
an external dc electric field is derived. Finally, a summary and
conclusions are presented in Sec. IV.


\vskip 1cm {\bf\large II. BASIC EQUATIONS }\vskip 0.5cm

We consider the effect of dust-charge variation on $DAW$ propagation
in a uniform, unmagnetized, partially ionized current-carrying dusty
plasma, whose constituents are negative electrons, positive ions,
micron-sized extremely massive negatively charged dust grains, and a
fraction of neutral atoms. For very low frequency waves such as
$DAW$s, the electron, ion, and dust grain temperatures satisfy $
T_{e}, T_{i} \gg T_{d}$ and the DA wave occurs in the frequency
range $kv_{Td} \ll \omega \ll kv_{Ti}, kv_{Te}$, where $k$ and
$\omega$ are wave number and frequency, respectively. $v_{Tj}$
denotes thermal velocity of the particle species $j$ where $j=e, i,
d$ for electrons, ions and dust particles, respectively. In the
absence of charge fluctuation the quasineutrality condition
$n_{i0}=n_{e0}+Z_{d0}n_{d0}$ is satisfied, where $n_{j0}$ is the
unperturbed number densities of the particle species $j$ and
$Z_{d0}$ represents the equilibrium number of charges residing on
the negatively charged dust grains. The dust particles are assumed
to be spherical and are of the same radius $a$. Charging of the dust
grains are considered to be connected to attachment of the electrons
and ions on the dust grains due to electron and ion currents
entering the dust grains.

For one-dimensional wave propagation along the x-axis, the dynamics
of plasma and dust particles are obtained from the fluid
equations.$^{15}$ For the plasma particles, we have
\begin{equation}\label{1}
\frac{\partial n_j}{\partial t}+\frac{\partial(n_{j}v_{j})}{\partial x}=-\nu_{jd}n_{j}+\zeta,
\end{equation}
\begin{equation}\label{2}
p v_{j}\frac{\partial v_{j}}{\partial x}+\nu_{j}^{eff}v_{j}+\frac{T_{j}}{n_{j}m_{j}}
\frac{\partial n_{j}}{\partial x}=\frac{q_{j}}{m_{j}}E,
\end{equation}
where $E$ is the electric field of the DAW, $n_{j}$ stands for the
sum of the equilibrium and the perturbed number densities of the
species $j$ (where $j = e, i$ for electron and ion respectively).
$v_{j}$, $q_{j}$ and $m_{j}$ denote fluid velocity, charge and mass
of the species $j$ respectively. $\nu_{jd}$ is the collection rate
of species $j$ by the dust grains. $p$ is equal to unity for ion
species and equal to zero for electrons. We have defined
$\zeta=\nu_{ion}n_{e}-\rho n_{e}^2$, where $\nu_{ion}$ is the
ionization rate, and $\rho$ is the volume recombination coefficient.
We have also defined the effective electron (ion) collision
frequency $\nu_{e}^{eff}=\nu_{e}+\nu_{e}^{el}+\nu_{e}^{ch}$
($\nu_{i}^{eff}=\nu_{i}+\nu_{i}^{el}+\nu_{i}^{ch}$) where $\nu_{e}$
($\nu_{i}$) is the rate of electron (ion) collisions with neutral
atoms and plasma particles, $\nu_{e}^{el}$ ($\nu_{i}^{el}$) is the
rate of elastic Coulomb electron-dust (ion-dust) collisions, and
$\nu_{e}^{ch}$ ($\nu_{i}^{ch}$) is the effective rate of collection
of electrons (ions) by the dusts.

For the dust particles, we have
\begin{equation}\label{3}
\frac{\partial n_d}{\partial t}+\frac{\partial(n_{d}v_{d})}{\partial x}=0,
\end{equation}
\begin{equation}\label{4}
\frac{\partial v_{d}}{\partial t}+v_{d}\frac{\partial v_{d}}{\partial x}
+\nu_{dn}v_{d}+\frac{T_{d}}{n_{d}m_{d}}\frac{\partial n_{d}}{\partial x}
+\mu_{drag}^{i}(v_{d}-v_{i})=-\frac{Z_{d}e}{m_{d}}E,
\end{equation}
where $n_{d}$ stands for the sum of the equilibrium and the
perturbed number density of the dust particle. $v_{d}$, $Z_{d}$ and
$m_{d}$ denote fluid velocity, charge number and mass of the dust
grain respectively.

In Eq. (\ref{4}) the term $\mu_{drag}^{i}(v_{d}-v_{i})$ corresponds
to the ion drag force acting on the dust grains, where
$\mu_{drag}^{i}\sim4m_{i}n_{0i}b^{2}v_{s}/m_{d}$ is the ion drag
coefficient. $v_{s}=\sqrt{T_{e}/m_{i}}$ is the ion-acoustic velocity
and $b\sim a\sqrt{\pi}(1-\Delta \varphi_{g}/T_{i})$ is the ion
collection impact parameter where $a$ is the radius of spherical
dust grains. $\Delta \varphi_{g}=(\phi_{g}-\phi_{0})=q_{d0}/C$
denotes the grain surface potential relative to the ambient plasma
potential where $q_{d0}$ is the equilibrium value of the dust charge
and $C=a(1+a/r_{D})$ is the grain capacitance. Clearly the ion drag
force, which is proportional to the square of particle radius,
dominates over the electrostatic force when dust grains have
sufficiently large sizes.$^1$

%In the unperturbed state, when the electric force is balanced by a ion drag force,
% the external electric field leads to a steady ion flow $v_{i0}=eE_{0}/m_{i}\nu_{i}^{eff}$.
%For electrons and dust grains in the presence of $E_{0}\^{x}$
%we have such equilibrium drift speeds but are not relevant for us.

The dust charge fluctuation is governed by the current balance
equation$^{1}$
\begin{equation}\label{5}
\frac{\partial {q}_{d1}}{\partial t}+v_{d}\frac{\partial {q}_{d1}}{\partial x}
+\nu_{d}^{ch}{q}_{d1}=-\vert I_{e0}\vert \frac{{n}_{e1}}{n_{e0}}
+\vert I_{i0}\vert\frac{{n}_{i1}}{n_{i0}}
\end{equation}
where ${q}_{d1}=Z_{d1}e$ is the perturbation of average charge,
$Z_{d1}$ is the perturbation of dust-charge number, ${n}_{e1}$
(${n}_{i1}$) is the variation of electron (ion) density.
$\nu_{d}^{ch}$ is the dust charging rate by the equilibrium electron
and ion microscopic currents.

Equation (\ref{5}) is valid for grain charging arising from plasma
currents due to electrons and ions reaching the grain surface. When
the streaming velocities of the electrons and ions are much smaller
than their corresponding thermal velocities and the thermal velocity
of the electrons are higher than the thermal velocity of the ions,
the surface potential of an isolated dust particle will be negative
and the electron and ion currents reaching to dust grains are
determined by
\begin{equation}\label{6}
I_{e0}=-\pi a^{2}e(8T_{e}/\pi m_{e})^{1/2}n_{e0}\exp \left[ e(\phi_{g}-\phi_{0})_{g}/T_{e}\right] ,
\end{equation}
\begin{equation}\label{7}
I_{i0}=-\pi a^{2}e(8T_{i}/\pi m_{i})^{1/2}n_{i0}\left[ 1-e(\phi_{g}-\phi_{0})/T_{i}\right] ,
\end{equation}

In the next section, using the dispersion relation of DAW and
considering that the ion drag force exceeds the electrostatic force,
we will investigate the filamentation instability, which takes place
at the initial stage of the formation of stable dust voids.

\vskip 1cm {\bf\large IV. DUST-ACOUSTIC FILAMENTATION }\vskip 0.5cm

By using appropriate expressions for collision frequencies,
considering quasi-neutrality condition for perturbed densities,
assuming monochromatic form for perturbed quantities and  then
linearizing Eqs (\ref{1}-\ref{5}), one can obtain dispersion
relation of $DAW$ for one-dimensional wave propagation along the
x-axis in aforementioned system as follow:$^{*****}$
$$\frac{\omega_{pe}^2}{\eta_{e}\nu_{e}^{eff}}\left( 1+\frac{\nu_{ed}}{\nu_{d}^{ch}}\right)
 +\frac{\omega_{pi}^2}{\eta_{i}\nu_{i}^{eff}}\left( 1+\frac{n_{e0}\nu_{ed}}{n_{i0}\nu_{d}^{ch}}\right)$$
\begin{equation}\label{8}
+\frac{i\omega_{pd}^{2}}{\Omega_{d0}[\Omega_{d0}+i(\nu_{dn}+\mu_{drag}^i)]}\left( 1-\frac{\Omega_{i0}+
i\nu_{id}}{\eta_{i}\beta}\right) =0
\end{equation}
where we have defined $\Omega_{d0}=\Omega-k v_{d0}$,
$\Omega_{j0}=\Omega_{d0}-k\vartheta_{j}$ and
$\vartheta_{j}=v_{j0}-v_{d0}$ for $j = e, i$.
$\beta=Z_{d0}m_{i}\nu_{i}^{eff}/m_{d}\mu_{drag}^{i}$ represents the
ratio of electric force to ion drag force and
$\omega_{p\alpha}=\sqrt{4\pi n_{\alpha}q_{\alpha}^2/m_{\alpha}}$ is
the plasma frequency of the $\alpha$ species.
$\eta_{i}=\Omega_{i0}+i\left(\nu_{id}+k^2V_{Ti}^{2}/\nu_{i}^{eff}\right)$
and $\sqrt{•}$ is the ion thermal velocity.

*********

*******

It is seen that the most unstable situations occur when the
percentage of free electrons is reduced.$^{16}$ Thus, we assume most
of background electrons are stick onto immobile dust grain surface
during charging processes. As a result, we have a significant
electron density depletion in dusty plasma, i.e., $n_{e0}\ll
n_{i0}$. In this case, Eq. (\ref{8}) reduces to
\begin{equation}\label{9}
\frac{\omega_{pi}^2}{\eta_{i}\nu_{i}^{eff}}+\frac{i\omega_{pd}^{2}}{\Omega_{d0}[\Omega_{d0}
+i(\nu_{dn}+\mu_{drag}^i)]}\left( 1-\frac{\Omega_{i0}+i\nu_{id}}{\eta_{i}\beta}\right) =0
\end{equation}
By invoking the quasi-neutrality approximation for negatively
charged dust grains ($Z_{d0}n_{d0}\approx n_{i0}$), Eq (\ref{2}) can
be rewritten into a simpler form
$$\frac{\Omega_{d}^3}{m_{i}}+i\frac{Z_{d0}k^{2}v_{Ti}^{2}}{m_{d}}\left( i\Omega_{d}
+\nu_{dn}+\mu_{drag}^{i}\right)+\frac{\mu_{drag}^i}{m_{i}} $$
\begin{equation}\label{10}
\times\left[ (\beta-1)(\Omega_{d}-k\vartheta_{i}+i\nu_{id})(i\Omega_{d}+\nu_{dn}+\mu_{drag}^{i})\right]=0
\end{equation}
where $v_{Ti}= \sqrt{T_{i}/m_{i}}$ is the ion thermal velocity.

We now consider the situation when the ion drag force acting on the
dust particles is stronger than the external electric force, that
is, when ${\beta}<1$. When the outward ion drag force exceeds inward
electrostatic force, the particles in that region repulse the
surrounding particles, until a stable circular void appears. In this
case, the dust grains are pushed out by the ion drag force and move
in the same direction with the ions. Thus, the relative ion-to-dust
drift is small and we can neglect of $k\vartheta_{i}$ in contrast
with $\nu_{id}$. Therefore, assuming $k\vartheta_{i}\ll \nu_{id}$,
we expand the dispersion equation (\ref{3}) in the static limit,
i.e., $\Omega_{d} \rightarrow 0$ to find the spatial structure
\begin{equation}\label{11}
k_{0}^{2}=\frac{m_{d}\mu_{drag}^{i}\nu_{id}(1-\beta)}{m_{i}Z_{d0}v_{Ti}^{2}}
\end{equation}
We can express the frequency spectrum in a weakly ionized
current-driven dusty plasma, as follows:
\begin{equation}\label{12}
\Omega_{d}=i\frac{m_{i}Z_{d0}k^{2}v_{Ti}^{2}}{m_{d}\mu_{drag}^{i}(1-\beta)}\left[ 1-\frac{k_{0}^2}{k^2}\right]
\end{equation}
This equation shows that the current-carrying dusty plasma is
unstable when $k_{0}^2<k^2$. Also, this spectrum shows that a
transverse structure with a characteristic period $\pi/k_0$ can
exist in the static limit and filamentation instability may arise
when $k_{0}^2<k^2$. As already mentioned, with reduction of ion drag
force the the filamentation will disappear, and the frequency
spectrum Eq. (\ref{5}) will become aperiodic. Also, this frequency
spectrum indicates that the filamentation instability will arise
when the particle diameter are sufficiently large. On the other
hand, when the dust grains are small the filamentation instability
thershold cannot be reached. According to Eq. (\ref{4}), the
parameters involved in $k_{0}^2$, the dimensions of filaments can be
high or low and the number of filaments dependent to plasma
dimensions and width filaments.

As we have a ion drift in the dusty plasma, a magnetic field
$B_{0}\approx 4\pi e n_{i}ux_{0}/c$ arises around its axis, where
$x_{0}$ is the lateral distance from current axis and $n_{i}$ is the
ion current density. If the transverse dusty plasma dimension is
greater than the the lateral distance from the current axis, the
magnetic pressure is larger than gas kinetic pressure and the
filamentation instability can be result i.e.,
\begin{equation}{\label{13}}
\frac{B_{0}^{2}}{8\pi}=\frac{1}{8\pi}\left[\frac{4\pi e n_{i} u
x_{0}}{c}\right]^{2} > n_{i}T_{i}
\end{equation}
where $n_{i}$ is the ion plasma density. From Eq. (\ref{4}) we can find
\begin{equation}{\label{14}}
x_{0} =\frac{2v_{Ti}\sqrt{m_{i}Z_{d0}}}{\sqrt{m_{d}\mu_{drag}^{i}\nu_{id}(1-\beta)}}\cong l_{0}
\end{equation}
Thus, the self-pressing (pinch effect) is possible.

The time needed for the establishment of this structure can be
determined from the time that is necessary for instability
development. To obtain this value we consider a system close to the
threshold $k\approx k_{0}$, where its minimal value can be
determined from Eq. (\ref{5}):
\begin{equation}\label{15}
\tau\cong \frac{1}{Im(\omega)}\approx\frac{m_{d}\mu_{drag}^{i}(1-\beta)}{m_{i}Z_{d0}k_0^{2}v_{Ti}^{2}}\approx\frac{1}{\nu_{id}}
\end{equation}
From Eq. (\ref{8}) we can find that the particle size or charge is a
critical parameter for the onset of the instability. Also,  the
establishment time of the filamentation instability is proportional
to the reverse of collision frequency between ion and dust
particles. This means that ion collisions have a destabilizing
effect on the dust acoustic waves.

\vskip 1cm {\bf\large V. NUMERICAL RESULTS }\vskip 0.5cm

In this section numerical solution of the dispersion relation Eq.
(\ref{5}) are obtained. In all of the results that follow we use the
following set of parameters:

...

For a typical dusty argon plasma,$^{15}$ the parameter values used
in the numerical computation are: $T_{i}\sim 0.15~ev$, $v_{Ti}\sim
5.7\times 10^4~cm/s$, $Z_{d0}m_{i}/m_d=1.95\times 10^{-6}$,
$\beta=0.01$, and $\mu_{drag}^i\sim1.65\times10^3~s^{-1}$. Hence, if
$\nu_{id}\sim10~s^{-1}$, we find $l_0\sim1.2~cm$ and
$\tau\sim~10^{-1}~s$.


\vskip 1cm {\bf\large IV. SUMMARY AND CONCLUSION }\vskip 0.5cm

In this work, using fluid theory and taken into account the dust
grain charge fluctuations, we investigated the fiilamentation
instability of dust-acoustic wave in a collisional dusty plasma. The
dust-acoustic wave was excited by a relative drift of the ions
produced by a steady-state electric field externally applied to the
dusty plasma. We assumed that most electrons are attached to the
immobile dust grains, i.e., $n_{i}\gg n_{e}$, because the dust
acoustic waves becomes more easily excited when the relative
concentration of negatively charged dust is increased. Then, we
obtained the dispersion relation for low-frequency dust-acoustic
wave and studied the filamentation instability in the aforementioned
system. We showed that this instability takes place when the ion
drag force acting on the dust particles is stronger than the
external electric force. In this case the ion drag force pushes the
dust grains in the same direction of  the ion drift. Thus, the
filamentary mode appeared abruptly in the initial stages of void
formation in dusty plasma. It was shown that if the dust particles
become small the electrostatic force will dominate, and the region
of redused dust density will be filled up once again by dust, and
the filamentation instability will disappear.  Also, it was shown
that close to a threshold, $k\cong k_{0}$, which dependence on the
particle size and the electric field strength, the current layer
will be subdivided into separate current filaments with a
establishment time of the order of $\tau  \approx 1/\nu_{id}$.

\newpage

\begin{thebibliography}{9}

$^{1}$ P. K. Shukla and A. A. Mamum, {\it Introduction to Dusty
Plasma Physics} (IOP Publishing, Bristul UK, 2002).
\\
$^{2}$ S. V. Vladimirov, K. Ostrikov and A. A. Samarian {\it Physics and Applications of Complex
Plasmas} (Imperial College Press, 2005).
\\
$^{3}$ V. E. Fortov, A. G. Khrapak, S. A. Khrapak, V. I. Molotkov
and O. F. Petrov, {\it Physics-Uspekhi} {\bf 47}, 447 (2004).
\\
$^{4}$ G. Praburam and J. Goree, {\it Phys. Plasmas} {\bf 3}, 1212 (1996).
\\
$^{5}$ D. Samsonov and J. Goree, {\it Phys. Rev. E} {\bf 59}, 1047 (1999).
\\
$^{6}$ G.E. Morfill, H. Thomas, U. Konopka, H. Rothermel, M. Zuzic,
A. Ivlev, J. Goree, {\it Phys. Rev. Lett.} {\bf83}, 1598 (1999).
\\
$^{7}$ E. Thomas, Jr., K. Avinash and R. L. Merlino, {\it Phys. Plasmas} {\bf 11}, 1770 (2004).
\\
$^{8}$ J. Goree, G.E. Morfill, V.N. Tsytovich, S.V. Vladimirov, {\it Phys.
Rev. E} {\bf 59}, 7055 (1999).
\\
$^{9}$ Z. Hu, Y. Chen, X. Zheng, F. Huang, G. Shi, and M. Y. Yu, {\it Phys. Plasmas} {\bf 16}, 063707 (2009).
\\
$^{10}$ X. Zheng, Y. Chen, Z. Hu, G. Wang, F Huang, C. Dong, and M. Y. Yu  {\it Phys. Plasmas} {\bf 16}, 023701 (2009).
\\
$^{11}$ A. A. Mamuni and P.K. Shukla {\it Phys. Plasmas} {\bf 11}, 1757 (2004).
\\
$^{12}$ A. A. Mamuni and P.K. Shukla {\it J. Plasma Phys} {\bf 71}, 143 (2005).
\\
$^{13}$ L. Couedel, M. Mikikian, A. A. Samarian and L. Boufendi {\it Phys. Plasmas} {\bf 17}, 083705 (2010).
\\
$^{14}$ A.A. Mamun, P.K. Shukla, R. Bingham {\it Phys. Lett. A} {\bf 298} 179, (2002).
\\
$^{15}$ K. N. Ostrikov, S. V. Vladimirov, M. Y. Yu and G. E.
Morfill, {\it Phys. Rev. E.} {\bf 61}, 4315 (2000)
\\
$^{16}$ N. D'Angelo and R. L. Merlino, {\it Planet. Space Sci.} {\bf
44}, 1593 (1996).
\\
$^{17}$ F. Califano, R. Prandi, F. Pegoraro, and S. V. Bulanov, {\it
Phys. Rev. E} {\bf 58}, 7837 (1998).
\\
$^{18}$ B. Shokri, S.M. Khorashadi and M. Dastmalchi, {\it Phys. of
Plamas}, {\bf 9}, 3355 (2002).
\\
$^{19}$ B. Shokri and T. Vazifehshenas, {\it Phys. Plasmas} {\bf 8},
788 (2001).
\\
$^{20}$ A. R. Niknam and B. Shokri {\it Phys. Plasmas} {\bf 15},
012108 (2008).
\\
$^{21}$ A. R. Niknam and B. Shokri {\it J. Plasma Phys.} {\bf 74},
319 (2008).
\\
$^{22}$ B. Mohammadhosseini, A. R. Niknam and B. Shokri {\it Phys. Plasmas} {\bf 17}, 122303
(2010).
\\
$^{23}$ F. Haas and P. K. Shukla, {\it Phys. Plasmas} {\bf 15},
093702 (2008).
\\
$^{24}$ S. M. Khorashadizadeh, A. R. Niknam and T. Haghtalab {\it Phys. of Plamas} {\bf 18}, 063703 (2011).
\\
$^{25}$ A. R. Niknam, T. Haghtalab and S. M. Khorashadizadeh {\it Phys. of Plamas} {\bf 18}, 113707 (2011).

\end{thebibliography}
\end{document}
